\section{Implementation}
\subsection{Restarting from an Equilibrium}
Once the first set of positively indexed equilibria has been
computed, the next step involves searching for other equilibria
from the already computed ones. This involves restarting from one of
the computed equilibrium, by choosing a covering vector and
pivoting in $z_0$. When an equilibrium has been computed,
the tableau column which represents the covering vector has
been changed and a new covering vector would need to be
computed which would be able to match the information in the
tableau.

The method used to compute the new covering vector is the
first method which is explained by Berhnard in his documentation 
of Lemke's algorithm \cite{doculemke}.
This involves the computation of
$A_B^{-1}$ for the set of basic variables in $B$, where $A$ is
the tableau which represents the current equilibrium. The column
$i$ in $A_B^{-1}$ is given by the column of $w_i$ if $w_i$ is 
non-basic, or by the column $x$ such that:
$x = 
\begin{pmatrix}
    0 & 0 & \cdots & det & \cdots & 0 & 0 &
\end{pmatrix}^\mathsf{T}$
where $det$ is the determinant of the tableau, and $x[l]$ is 
equal to $det$ if $l$ is the row of $w_i$, or 0 otherwise.

The covering vector $d$ which represents a missing label $k$ is
given by 
\[d = 
\begin{pmatrix}
    1 & 1 & \cdots & 0 & \cdots & 1 & 1 &
\end{pmatrix}^\mathsf{T}\]
such that $d[i] = 0$ if $i = k$ otherwise $d[i] = 0$. Once the
values of $d$ and $A_B^{-1}$ have been computed, the new
covering vector $d'$ is calculated as $A_B^{-1}d$. The new
covering vector $d'$ can then be substituted in the tableau for
the column for $z_0$.

\subsection{Bi-partite Graph}
The bi-partite graph containing the equilibria of a game is 
represented in the program by two linked list, one containing
all the negatively indexed equilibria, and the second has as
its elements all the positively indexed equilibria. Each node in
a linked list stores three values; the equilibrium represented
at that node, the list of equilibria the current equilibria is
connected with a given missing label, and the next element in
the list. By storing each partition of the graph as a list,
we can easily get the $i$-th equilibrium in a list containing
$n$ elements, where $0 \leq i < n$, by using the value for
next in each node until the $i$-th node has been found.

Each equilibrium found is stored as the tableau which represents
the equilibrium. By doing so the equilibrium for the given
node can be copied into the variables used for computation of the
new equilibrium.

The list of equilibria which can be gotten from the current equilibria
using a set of missing labels is stored as an integer array \verb|link|
where the $k$-th element of the array $l$, is the location of the
equilibrium the current equilibrium is linked to in the opposite
list by the missing label $k-1$.

The bi-partite graph is computed using an algorithm similar to the 
Breadth First Search (BFS) to compute all equilibria reachable from
the artificial equilibrium which is listed in Algorithm \ref{compall}.
\begin{algorithm}
    \caption{Compute Equilibria From Node}\label{compnode}
    \begin{algorithmic}[1]
        \Procedure{Compute Equilibria From Node}{$i, isneg$}\Comment{\small{Computes all equilibria from the current node $i$}}
            \If{$isneg$}
                \State $cur \gets neg$ \Comment{Set current list as the negatively indexed list}
                \State $res \gets pos$ \Comment{Set result list as the positively indexed list}
            \Else
                \State $cur \gets pos$ \Comment{Set current list as the positively indexed list}
                \State $res \gets neg$ \Comment{Set result list as the negatively indexed list}
            \EndIf
            \\
            \State $cur \gets getNodeAt(cur, i)$ \Comment{Gets the $i$th node in the list}
            \State $maxK \gets nrows + ncols$ \Comment{The maximum possible value for the missing label}
            \\
            \For{$k2 \gets 1, maxK$}
                \If{$cur.link[k2-1] \not= -1$} \Comment{If the equilibrium it is linked with using label $k2$ is known}
                    \State \textbf{continue}
                \EndIf
                \\
                \State $copyEquilibrium(cur.eq)$ \Comment{Copy and setup the current equilibrium for computation}
                \State $setupEquilibrium()$
                \State $runLemke()$ \Comment{Run Lemke's algorithm to find the next equilibrium}
                \\
                \State $eq \gets createEquilibrium()$ \Comment{Create equilibrium using the current information in the tableau}
                \State $j \gets addEquilibrium(res, eq)$ \Comment{Add the equilibrium to $res$ and return its location. If
                $eq$ already exists in $res$, it returns its location but doesn't add it to $res$}
                \\
                \State $cur.link[k2-1] \gets j$ \Comment{Link the two equilibria together using label $k2$}
                \State $p \gets getNodeat(res, j)$
                \State $p.link[k2-1] \gets i$
            \EndFor
        \EndProcedure
    \end{algorithmic}
\end{algorithm}

\begin{algorithm}
    \caption{Compute all Equilibrium Reachable from the Artificial Equilibrium}\label{compall}
    \begin{algorithmic}[1]
        \Procedure{Compute All Equilibria}{} \Comment{\small{}}
            \State $maxK \gets nrows + ncols$
            \State $neg = newnode(maxK)$ \Comment{Create the two lists with the total number of strategies}
            \State $pos = newnode(maxK)$
            \State $isqdok()$ \Comment{Check if the LCP is valid and generate the tableau}
            \State $filltableau()$
            \\
            \State $eq \gets createEquilibrium()$ \Comment{Create the Artificial Equilibrium from the current tableau}
            \State $neg.eq \gets eq$ \Comment{Set $neg[0]$ to the Artificial Equilibrium}
            \State $runLemke()$ \Comment{Run Lemke's algorithm to find the first positively indexed Equilibrium}
            \State $eq \gets createEquilibrium()$
            \\
            \State $pos.eq \gets eq$
            \State $neg.link[0] \gets 0$ \Comment{Link the two equilibria using label 1}
            \State $pos.link[0] \gets 0$
            \\
            \State $negi \gets 1$ \Comment{Stores its location in the list of negatively indexed Equilibria}
            \State $posi \gets 1$ \Comment{Stores its location in the list of positively indexed Equilibria}
            \State $isneg \gets \textbf{FALSE}$ \Comment{Stores if it is currently restarting from a negatively indexed Equilibrium}
            \\
            \While{\textbf{TRUE}}
                \If{$isneg$}
                    \While{$negi < listlength(neg)$}
                        \State $computeEquilibriaFromNode(negi, isneg)$
                        \State $negi$++
                    \EndWhile
                \Else
                    \While{$posi < listlength(pos)$}
                        \State $computeEquilibriaFromNode(posi, isneg)$
                        \State $posi$++
                    \EndWhile
                \EndIf
                \State $isneg = !isneg$
                \\
                \If{$(negi == listlength(neg))$ \& $(posi == listlength(pos))$}
                    \State \textbf{break}
                \EndIf
            \EndWhile
        \EndProcedure
    \end{algorithmic}
\end{algorithm}

Starting from the artificial equilibrium which is stored in \verb|neg[0]|,
it first computes the first positively indexed equilibrium by using the
missing label 1, stores the resulting equilibrium in \verb|pos[0]| and
and links both nodes by setting the first element of both links to 0
(i.e. \verb|neg->link[0] = 0| and \verb|pos->link[0] = 0|). It then makes
a call to compute all the equilibrium from the first node of the negatively
indexed equilibria (i.e. the artificial equilibrium). Once all the first set
of positively linked equilibria have been calculated, it sets \verb|negi| to
1 and \verb|posi| to 0. The value of \verb|negi| is used to store the location
of the next negatively indexed equilibrium to restart from, the same goes for
\verb|posi| but for positively indexed equilibrium, \verb|isneg| is also set to
FALSE (0), to represent that we are going to be restarting from a positively indexed
equilibrium.

The algorithm then loops through the following steps until the condition
that \verb|negi| equals the number of negatively indexed equilibria, and
\verb|posi| equals the number of positively indexed equilibria, indicating
that there are no longer any equilibria which can be found.found.

If \verb|isneg| is FALSE, it loops through values of \verb|posi| until \verb|posi|
is equal to the length of \verb|pos|. For each value of \verb|posi| it computes
all the negatively indexed equilibria which are reachable from the \verb|posi|-th 
positively indexed equilibrium, storing any new equilibrium found to \verb|neg|
and linking the current equilibrium with the equilibrium found using the label it
was found with. After each step, \verb|posi| is incremented by 1.found.

If \verb|isneg| is TRUE, the process is the same as the above paragraph, but
with \verb|negi| instead of \verb|posi|, \verb|neg| instead of \verb|pos| and
\verb|pos| instead of \verb|neg|.

Upon completion of the loop, the values of \verb|negi| and \verb|posi| would
store the number of negatively indexed and positively indexed equilibria found
respectively.