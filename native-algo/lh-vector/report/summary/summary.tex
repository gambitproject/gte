% 26 May 2012
\documentclass[a4paper,12pt]{article}  %% important: a4paper first
% \pdfoutput=1
\usepackage{mathptmx}
\usepackage{epsf}
\usepackage{amssymb}
\usepackage{amsmath}
\usepackage{bbold}
\usepackage{graphicx}
\usepackage{hyperref}
\usepackage{algorithm}
\usepackage{algpseudocode}
\usepackage{asymptote}
\usepackage{pict2e}
\usepackage{xcolor}

\oddsidemargin=.46cm    % A4
\textwidth=15cm
\textheight=23.3cm
\topmargin=-1.3cm
\clubpenalty=10000
\widowpenalty=10000
\predisplaypenalty=1350
\newdimen\einr
\einr2em

\setlength\fboxrule{0pt}

% \parindent0pt
\parskip1ex

\title{Summary Report}

\author{Tobenna Peter, Igwe\\ \small{\url{http://www.github.com/ptigwe/lh-vector}}}

\date{\today}

\begin{document}
    \maketitle
    \section{Introduction}
    Through out the course of this project, the computation of all equilibria reachable
    from the artificial equilibrium with the Lemke-Howson algorithm using covering
    vectors has been implemented. The project was built upon Berhnard von Stengel's
    implementation of Lemke's algorithm in Linear Complementarity Problems (LCP) from
    \url{http://www.github.com/stengel/ecta2002}. This
    report provides a brief summary of how system that was implemented, and some of
    the observations which were made.
    
    \section{Implementation}
    Minor changes were made to the underlying implementation of Lemke's algorithm
    which were mainly the way it represented rational numbers. The functionality
    of computing rational numbers having the numerator and denominator integers
    of large numbers was added using the GNUMP library as well as the multiple
    precision (\verb|mp.h|) files provided by Berhnard.
    
    The algorithm used to navigate the bi-partite graph of equilibria, is similar
    to the Breadth First Search(BFS) algorithm. Starting from the artificial
    equilibrium, for each missing label $k$ in ${1 \cdots m+n}$ in a given
    $m \times n$ game, it computes the equilibrium by computing the
    value of the covering vector $d$ which represents the missing label and
    following a set of pivoting rules. All computed equilibrium are stored as nodes
    in the bi-partite graph, and the labels are used as the edges linking the two
    equilibria (i.e the equilibrium started from and the computed equilibrium).
    For every equilibrium $e$ found, the algorithm then tries to compute the
    equilibria which can be found from $e$. When restarting from $e$, for all
    missing labels $k'$ in ${1 \cdots m+n}$, if there is an an edge from $e$ to
    an already computed equilibrium using $k'$, then restarting from $e$ using the
    missing label $k'$ is skipped to avoid backward computation. Otherwise, the
    covering vector $d'$ which represents $k'$ is computed and inserted into the
    tableau, $z_0$ is pivoted, and Lemke's algorithm finds a new equilibrium, $e'$
    which is added to the graph and linked with $e$ using label $k'$. Note that
    all equilibrium in the first set of equilibrium are restarted from before any
    equilibrium in the second set, and so on.
    
    The algorithm terminates when there is a given set of equilibrium such that
    no new equilibrium can be computed from it (i.e. when restarting from such
    an equilibrium, for all missing labels $k'$, there exists a link from the
    equilibrium to an already computed equilibrium).
    
    \section{Testing and Observations}
    The test cases used can be found in the folder \verb|test|. These include
    identity matrix games and dual-cyclic polytope games. It also includes testing
    rational and decimal input system. A bash script which runs all the tests is
    available at \verb|test/testall|, and accepts one argument, which is the
    size of the identity matrix game to be tested. Some other test cases were
    done using best response diagrams and can be found in the final report 
    (\url{http://www.csc.liv.ac.uk/~cs0tpi/GSOC/final.pdf}).
    
    While testing some of the games, it was noticed that in some cases the 
    path length ended up being had 4 extra pivots instead of 3 extra pivots to
    the LH-path. This was mainly noticed with $\Gamma(d \times 2d)$ dual-cyclic
    polytope games, for missing labels whose path lengths were odd numbers.
    
    \section{Input and Output}
    \subsection{Input}
    The program receives as its input a 2-player bi-matrix game in the format\\
    \[
    \begin{matrix}
        m= & ?\\
        n= & ?\\
        A=\\
        &\ddots\\
        B=\\
        &\ddots\\
    \end{matrix}\]
    where $A$ and $B$ are the payoff matrices for player 1 and 2 respectively.
    For example, for a $2 \times 2$ identity matrix game, the input would be:
    \[
    \begin{matrix}
        m= & 2\\
        n= & 2\\
        A=\\
        1 & 0\\
        0 & 1\\
        B=\\
        1 & 0\\
        0 & 1\\
    \end{matrix}
    \]
    \subsection{Output}
    The output provided can be split into two parts, the LH-path and the
    bi-partite graph.
    
    As the program is computing the path from a given equilibrium, it
    first prints out the tableau representing the current equilibrium
    which it is starting from, followed by the set of pivot steps, and
    finally the tableau representing the current equilibrium. If the
    program is run with the verbose flag (\verb|-v|), it prints out
    the tableau after every pivot step.
    
    Following the information of the LH-path, is the information of the
    bipartite graph. This is of the following format:
    \flushleft{
    \[
    \begin{matrix}
        \textsf{Equilibria discovered:} & x\\
        \textsf{y-ve index:}\\
        \textsf{Eq[0]:}& \cdots\\
        \textsf{Leads to:} & \cdots k \verb|->| i \cdots\\
        \vdots\\
        \textsf{Eq[y-1]:} &\cdots\\
        \textsf{Leads to:} & \cdots k \verb|->| i \cdots\\
        \\
        \textsf{z-ve index:}\\
        \textsf{Eq[0]:}& \cdots\\
        \textsf{Leads to:} & \cdots k \verb|->| i \cdots\\
        \vdots\\
        \textsf{Eq[z-1]:} &\cdots\\
        \textsf{Leads to:} & \cdots k \verb|->| i \cdots\\
    \end{matrix}
    \]}
    where the program found $x$ equilibria excluding the artificial equilibrium,
    of which $y$ have a negative index, and $z$ have a positive index. For each 
    equilibrium, the strategy is printed out in form of fractions, containing both
    player 1 and 2's strategies. It also shows how the equilibria are linked together,
    and for every value of the missing label $k$, it shows the current equilibrium
    is linked to some other equilibrium with location $i$ in the opposite list. For
    example, if we had $2 \verb|->| 1$, and the current equilibrium is \verb|Eq[2]|,
    and it is negatively indexed, then the missing label 2 would lead us to the equilibrium
    \verb|Eq[1]|. 
    
\end{document}